% ! TeX root = thesis.tex
\begin{abstract}
    Questa tesi nasce dalla necessità di sviluppare un software anti plagio per l'analisi dei progetti d'esame del corso di Programmazione ad Oggetti dell'Università di Bologna.
    L'elaborato, quindi, si addentra nelle tecniche di analisi e individuazione di possibili plagi e presenta il processo di progettazione e sviluppo dello strumento.
    La tesi è strutturata in capitoli: nel primo vengono brevemente introdotti i problemi da affrontare durante lo sviluppo di strumenti anti plagio; nel secondo si descrivono le principali tecniche presenti in letteratura. Nel terzo e quarto capitolo ci si addentra nell'analisi dei requisiti e nel processo di implementazione. Un ultimo capitolo è dedicato al \textit{testing} e all'analisi delle prestazioni ottenute.
\end{abstract}