% ! TeX root = ../../thesis.tex
\begin{abstract}
    Questa tesi nasce dalla necessità di sviluppare uno strumento capace di individuare potenziali plagi in progetti \textit{software}.
    %
    Il plagiarismo nel software è pratica che, nel tempo, ha visto numerosi scontri legali (ad esempio, Oracle contro Google per Android), e per il quale sono pochi i progetti \textit{open source} di facile utilizzo pratico.
    %
    L'elaborato, quindi, si addentra nelle tecniche di analisi e d'individuazione di possibili plagi, presentando il processo di progettazione e sviluppo dello strumento.

    \vspace*{0.3cm}

    La tesi è strutturata in capitoli. 
    %
    Nel primo viene introdotto il contesto, vengono esposti i problemi da affrontare durante lo sviluppo di strumenti antiplagio e viene presentata una panoramica delle principali tecniche presenti in letteratura. 
    %
    Nel secondo e terzo capitolo ci si addentra nell'analisi dei requisiti, nella progettazione e nel processo d'implementazione dello strumento. 
    %
    Un ultimo capitolo è dedicato all'analisi dei risultati ottenuti.
\end{abstract}