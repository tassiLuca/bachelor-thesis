% ! TeX root = ../../thesis.tex
\chapter{Validazione, conclusioni e lavori futuri}
\label{chapter:validation}
In questo capitolo vengono riportati e analizzati i risultati ottenuti nell'ambito di applicazione dello strumento sviluppato.
%
Per ultimo, si traggono le conclusioni del progetto, elencando brevemente i possibili sviluppi futuri.

\section{Valutazione qualitativa}
Le prestazioni del sistema sono misurate sulla base di due aspetti fondamentali e indipendenti tra loro: le \textit{performance}, in termini di tempo d'esecuzione, ma soprattutto l'accuratezza.
%
Infatti, lo scopo primario è rilevare con precisione casi probabili di plagio e presentarli all'utente il più in alto possibile nei \textit{report} generati dallo strumento.
%
\`E solo in seguito ad aver raggiunto un grado di precisione soddisfacibile che possono essere adottate ottimizzazioni che permettano di ridurre il tempo di calcolo.

Dunque, per testarne l'accuratezza sono stati considerati 130 diversi progetti del corso di Programmazione ad Oggetti presentati nell'arco degli ultimi tre anni accademici e per ciascuno di questi si è effettuato il confronto con tutti i progetti a disposizione dal 2015 in poi che, complessivamente, ammontano a 354 progetti.
%
Questi sono sviluppati in Java, generalmente in \textit{team} di quattro componenti, e rappresentano artefatti di medio-alta complessità, costituiti in media da più di un centinaio di sorgenti ciascuno.

\subsection{Analisi di sensitività}
Dapprima si è testato il sistema eseguendo una scansione senza applicare alcun filtraggio intermedio delle rappresentazioni come descritto nella \Cref{03:filtering-phase}, scegliendo come parametri di riferimento i seguenti:
\begin{itemize}
    \item $s$, minima lunghezza della sequenza di \textit{token} rilevabile dall'algoritmo di \textit{detection} (\textit{RKR-GST}), pari a 15;
    \item valore di soglia percentuale sotto la quale la similarità di una coppia di sorgenti non viene segnalata pari al 30\%;
    \item metrica di similarià sorgente-sorgente espressa dall'\Cref{eq:avg-norm-sim}.
\end{itemize}

A seguito dell'analisi si è effettuata un'ispezione manuale dei sorgenti in modo tale da verificare che le stime effettuate dallo strumento non fossero fuorvianti o prive di fondamento.

I risultati complessivi sono presentati in \Cref{table:results}; per semplicità sono riportati solo le corrispondenze con similarità maggiore del 40\%.

\begin{table}[h!]
    \centering
    \begin{tabular}{|p{0.2\linewidth}|p{0.2\linewidth}|p{0.15\linewidth}|p{0.35\textwidth}|}
        \hline
        \textbf{Progetto originale} & \textbf{Progetto copiato} & \textbf{Similarità} & \textbf{Ispezione manuale} \\ [0.5ex] 
        \hline\hline
        gym-man & gym-man & 100\% & corrispondenza totale \\
        \hline
        spacewar & Space-Invaders & 90\% & corrispondenza totale \\
        \hline
        zombiegame & boxhead & 81\% & corrispondenza elevata \\
        \hline
        paranoid & b.b.evo & 70\% & corrispondenza medio-alta \\
        \hline
        space-run & alone-in-the-space & 62\% & corrispondenza medio-alta \\
        \hline
        Flappy Bird & SMR & 55\% & corrispondenza parziale \\
        \hline
        Teatro & harbor-mgr & 49\% & corrispondenza parziale \\
        \hline
        ALA & traffic-sim & 48\% & nessuna corrispondenza - "falso positivo" \\
        \hline
    \end{tabular}
    \caption[Risultati ottenuti confrontando i progetti sottomessi negli ultimi tre anni accademici]{Risultati ottenuti confrontando i progetti consegnati negli ultimi tre anni accademici. Nell'ultima colonna vi è riportato l'esito dell'ispezione manuale eseguita a seguito di quella automatica; per semplicità si sono catalogate le corrispondenze in: corrispondenza \textit{totale}, \textit{elevata}, \textit{parziale} o \textit{falso positivo}.}
    \label{table:results}
\end{table}

Come si può osservare, lo strumento è in grado di rilevare molto bene copiature evidenti, ovvero casi in cui sono state copiate intere classi di sorgenti: è il caso dei primi due risultati in cui i progetti sono pressoché identici e la percentuale di similarità stimata non è inferiore al 90\%.
%
%Nel secondo caso la similarità al 90\% è spiegabile dal fatto che alcune interfacce sono molto piccole e non hanno una lunghezza (in numero di \textit{token}) sufficiente per essere rilevate (cioè sono minori della minima lunghezza $s$ imposta nell'algoritmo \textit{Running Karp Rabin Greedy String Tiling}).

Come naturale che sia, invece, laddove la copiatura è effettuata in modo più scaltro, attingendo in modo parziale da altri progetti, la similarità cala, mantenendosi tuttavia al di sopra del 60\%.

Sotto il 60\% si è in presenza di codice piuttosto simile, ma in maniera molto limitata rispetto alla totalità dei sorgenti.
%
\`E il caso, ad esempio, dei progetti \texttt{SMR} e \texttt{Flappy Bird} in cui è intuibile, guardando i risultati generati dal rilevatore, che gli sviluppatori del secondo abbiano preso spunto dal primo.
%
Tuttavia non è ravvisabile una copiatura evidente come nei precedenti casi.

Un solo caso, tra tutti i progetti analizzati, riporta similarità maggiore del 45\% senza essere realmente copiato.
%
Ciononostante, la similarità riportata non è superiore al 50\% e il fatto che i progetti non siano copiati né simili è facilmente verificabile in pochi minuti guardando il \textit{report} stesso, che comprende solo costanti e \textit{getter/setter}.

Successivamente, sono state rieffettuate le scansioni dei soli progetti somiglianti presentati in \Cref{table:results}, variando in modo opportuno i parametri scelti al fine d'identificare la loro combinazione ottimale da utilizzare.

Come si può notare dalla \Cref{table:grid-search} le misurazioni di similarità più accurate in relazione all'ispezione manuale riportata in \Cref{table:results} sono in corrispondenza di una lunghezza $s$ variabile tra 15 e 20 e un valore di soglia di duplicazione minima compreso tra il 30 e il 40\%.
%
Infatti, scegliere una lunghezza $s$ e un valore di duplicazione al di sopra di questi limiti, seppur mantenga inalterate, se non migliori, le prestazioni per i casi di copiature evidenti (è questo il caso della coppia di progetti \texttt{Spacewar} e \texttt{space-invaders}), diminuisce l'accuratezza nei casi in cui la copiatura è meno evidente.

Inoltre, si noti che la misurazione della similarità mediante la normalizzazione massima, così come descritta nell'\Cref{eq:max-norm-sim}, in concomitanza con quella dei progetti (\Cref{eq:project-similarity}) genera risultati fuorvianti in corrispondenza di piccoli progetti e/o sorgenti di piccola dimensione.
%
Questo fenomeno è facilmente osservabile dalle stime di similarità tra i progetti \texttt{Flappy-bird} e \texttt{SMR}. 
%
Questo ci spinge a preferire l'utilizzo della metrica di normalizzazione media descritta nell'\Cref{eq:max-norm-sim} che fornisce risultati più stabili, derivanti dal fatto che considera la media della lunghezza dei sorgenti e non il sorgente più corto per il calcolo della similarità.

Dunque, la combinazioni di parametri che forniscono risultati migliori secondo i \textit{test} effettuati sono le seguenti:

\begin{multicols}{2}
    \begin{itemize}
        \item $s = 15$;
        \item $min-dup=0.3$
        \item $metrica=avg$
    \end{itemize}
    
    \begin{itemize}
        \item $s = 15$;
        \item $min-dup=0.4$
        \item $metrica=avg$
    \end{itemize}
\end{multicols}

Entrambi sono due scelte ragionevoli: la seconda offre il vantaggio di annullare il falso positivo, portando la sua stima di similarità ad essere inferore del 25\%, ma, d'altro canto, mette in maggior risalto (in maniera ingiustificata) le corrispondenze tra \texttt{Flappy-bird/SMR} e \texttt{Teatro/harbor-mgr} che, presumibilmente, ci si aspetterebbe sotto il 60\%, mentre la prima fornisce una stima più accurata della similarità di questi ultimi casi, ma presenta lo svantaggio di riportare un falso positivo all'attenzione dell'utente.

\begin{landscape}
    \begin{table}[h!]
        \centering
        \begin{tabular}{ p{0.5cm}|c|c|c||p{1cm}|p{1.5cm}|p{1.5cm}|p{1.5cm}|p{1.5cm}|p{1.2cm}|p{1.2cm}|p{1.2cm} }
            \multicolumn{4}{c}{} & \multicolumn{8}{c}{\textbf{Coppie di progetti simili}} \\ [1ex]
            \multicolumn{4}{c|}{} & gym-man $-$ gym-man & spacewar $-$ space-invaders & zombie-game $-$ boxhead & paranoid $-$ b.b.evo & Space-run $-$ alone-in-the-space & flappy-bird $-$ SMR & Teatro $-$ harbor-mgr & ALA $-$ traffic-sim \\ [1ex]
            \hline\hline

            \multirow{18}{*}{\begin{turn}{90}\textbf{parametri}\end{turn}} & \multirow{6}{*}{$\text{min-dup}=0.3$} & \multirow{2}{*}{$s=15$} & Max & 100\% & 95\% & 72\% & 72\% & 75\% & 80\% & 56\% & 74\% \\
            & & & Avg & 100\% & 90\% & 81\% & 70\% & 62\% & 55\% & 49\% & 48\% \\
            & & \multirow{2}{*}{$s=20$} & Max & 100\% & 99\% & 72\% & 75\% & 63\% & 83\% & 50\% & 57\% \\
            & & & Avg & 100\% & 91\% & 58\% & 71\% & 48\% & 61\% & 51\% & 31\% \\
            & & \multirow{2}{*}{$s=25$} & Max & 100\% & 98\% & 54\% & 66\% & 49\% & 79\% & 46\% & 26\% \\
            & & & Avg & 100\% & 92\% & 42\% & 58\% & 41\% & 51\% & 40\% & 12\% \\
            \cline{2-12}

            & \multirow{6}{*}{$\text{min-dup}=0.4$} & \multirow{2}{*}{$s=15$} & Max & 100\% & 96\% & 81\% & 76\% & 70\% & 79\% & 61\% & 75\% \\
            & & & Avg & 100\% & 92\% & 73\% & 79\% & 57\% & 62\% & 55\% & 22\% \\
            & & \multirow{2}{*}{$s=20$} & Max & 100\% & 99\% & 65\% & 85\% & 63\% & 80\% & 57\% & 47\% \\
            & & & Avg & 100\% & 97\% & 49\% & 69\% & 44\% & 54\% & 39\% & 10\% \\
            & & \multirow{2}{*}{$s=25$} & Max & 100\% & 99\% & 49\% & 64\% & 42\% & 80\% & 46\% & 64\% \\
            & & & Avg & 100\% & 99\% & 36\% & 62\% & 34\% & 52\% & 20\% & 6\% \\
            \cline{2-12}

            & \multirow{6}{*}{$\text{min-dup}=0.5$} & \multirow{2}{*}{$s=15$} & Max & 100\% & 99\% & 80\% & 84\% & 64\% & 76\% & 67\% & 27\% \\
            & & & Avg & 100\% & 99\% & 51\% & 73\% & 33\% & 52\% & 33\% & 14\% \\
            & & \multirow{2}{*}{$s=20$} & Max & 100\% & 99\% & 56\% & 82\% & 59\% & 56\% & 45\% & 35\% \\
            & & & Avg & 100\% & 99\% & 30\% & 67\% & 30\% & 47\% & 23\% & 4\% \\
            & & \multirow{2}{*}{$s=25$} & Max & 100\% & 99\% & 44\% & 63\% & 38\% & 56\% & 27\% & 15\% \\
            & & & Avg & 100\% & 98\% & 21\% & 55\% & 27\% & 47\% & 11\% & 3\% \\
            \hline
        \end{tabular}
        \caption[]{}
        \label{table:grid-search}
    \end{table}
\end{landscape}

Si è infine testato il risultato dell'analisi eseguendo il filtraggio delle rappresentazioni come descritto nella \Cref{03:filtering-phase}.

\begin{table}[h!]
    \centering
    \begin{tabular}{|c|c|c|c|c|}
        \hline
        \diagbox{\textbf{Progetti}}{\textbf{Soglia filtro}} & \textbf{assente} & \textbf{0,3} & \textbf{0,4} & \textbf{0,5} \\ [0.5ex] 
        \hline\hline
        gym-man - gym-man & 100\% & 100\% & 100\% & 100\% \\
        \hline
        spacewar - Space-Invaders & 90\% & 90\% & 90\% & 90\% \\
        \hline
        zombiegame - boxhead & 81\% & 81\% & 81\% & 80\% \\
        \hline
        paranoid - b.b.evo & 70\% & 70\% & 70\% & 78\% \\
        \hline
        space-run - alone-in-the-space & 62\% & 62\% & 62\% & 59\% \\
        \hline
        Flappy Bird - SMR & 55\% & 55\% & 55\% & 55\% \\
        \hline
        Teatro - harbor-mgr & 49\% & 49\% & 49\% & 49\% \\
        \hline
        ALA - traffic-sim & 48\% & 48\% & 48\% & 46\% \\
        \hline
    \end{tabular}
    \caption{}
    \label{table:filter-results}
\end{table}

Osservando i risultati, si conviene che l'utilizzo di una fase intermedia di filtraggio di rappresentazioni non inficia le stime similarità dei casi presentati se il valore di taglio è mantenuto al di sotto del 50\%.

\vspace*{0.3cm}

In definitiva, dai risultati sopra riportati, una \textbf{buona soglia di "sospetto"} sopra la quale l'utente dovrebbe concentrare la sua attenzione è stimabile sul \textbf{45\%}, sapendo che valori più alti di similarità implicano maggiore evidenza nella copiatura.

\subsection{Analisi di sensibilità}
Dedichiamo ora qualche osservazione sulla sensibilità dello strumento.

Come sappiamo, il processo di \textit{tokenizzazione} ci permette di convertire codice sorgente in una sequenza di \textit{token} che, da un lato, astragga il più possibile dalla sintassi del linguaggio, e dall'altra che lo descriva a livello semantico, cosicché possano essere riconosciute sequenze simili corrispondenti a porzioni simili di codice a livello semantico.

Quando, tuttavia, si opera un'azione di astrazione di questo tipo si va incontro al rischio che strutture simili di sorgenti, e tuttavia diverse semanticamente, vengano convertite in sequenze di \textit{token} simili, generando similarità fuorvianti.

Questo caso si verifica, ad esempio, con interfacce o classi molto semplici dove sono presenti solo \textit{getter} e \textit{setter} o con una sequenza di costanti.
%
A riprova di questo, vengono riportate di seguito (\Cref{code:04-false-positive-enum} e \Cref{04-false-positive-simple-class}) alcune sezioni che il rilevatore ha determinato essere simili nella coppia di progetti riconosciuta come un caso di falso positivo, del paragrafo precedente. 

\lstinputlisting[
    caption={Esempio di falso positivo dovuto alla presenza di molteplici costanti},
    label={code:04-false-positive-enum}
]{resources/code/04-false-positive-enum}

\newpage

\lstinputlisting[
    caption={Esempio di falso positivo dovuto alla presenza di una sequenza molto simile di \textit{getter/setter}},
    label={04-false-positive-simple-class}
]{resources/code/04-false-positive-simple-class}

Come si può osservare entrambi i sorgenti sono erroneamente classificati come somiglianti con un'elevata percentuale, superiore al 60\%.
%
Quando questo fenomeno è limitato a qualche istanza, ciò non comporta un particolare problema in quanto viene assorbito e bilanciato dalle diverse metriche che computano una stima media delle similarità.
%
Laddove, invece, il numero di queste errate classificazioni cresce in modo corposo,  si possono verificare casi di falsi negativi come quello presentato nel paragrafo precedente.

Si evidenzia, tuttavia, che questi casi di errata classificazione \textit{non} si verificano in presenza di sorgenti con un grado di logica rilevante, dove la sequenza di \textit{token} corrispondono se effettivamente la logica è molto simile.
%
Perdipiù, questi casi sono facilmente individuabili mediante un'ispezione manuale e non rappresentano un elevato rischio in termini di accuratezza.

\section{Tempi d'esecuzione}
Per quanto concerne i tempi d'esecuzione, attualmente lo strumento impiega, per fare il confronto di un progetto con i 354 progetti disponibili, grazie alla parallelizzazione e al salvataggio in locale dei sorgenti da analizzare, mediamente un terzo del tempo che avrebbe impiegato senza effettuare il \textit{caching}, stimabile all'incirca in al più 20 minuti.
%
Questo dipende anche in maniera incontrollabile dal calcolatore su cui viene eseguito.

\section{Conclusioni}

\section{Sviluppi futuri}
