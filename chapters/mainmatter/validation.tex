% ! TeX root = ../../thesis.tex
\chapter{Validazione dei risultati}
\label{chapter:validation}
In questo capitolo vengono riportati e analizzati i risultati ottenuti nell'ambito di applicazione dello strumento sviluppato.

\vspace*{0.5cm}

Le prestazioni del sistema sono misurate sulla base di due aspetti fondamentali e indipendenti tra loro: le \textit{performance}, in termini di tempo d'esecuzione ma, soprattutto, l'accuratezza, ovvero la capacità di rilevare copiature e plagi.

\section{Accuratezza}
L'accuratezza è l'obiettivo più importante da tenere in considerazione e in relazione alla quale misurare le prestazioni.
%
Infatti, lo scopo primario è rilevare con precisione casi probabili di plagio e presentarli all'utente il più in alto possibile nei \textit{report} generati dallo strumento.
%
\`E solo in seguito ad aver raggiunto un grado di precisione soddisfacibile che possono essere adottate ottimizzazioni che permettano di ridurre il tempo di calcolo.

Dunque, per testarne l'accuratezza sono stati considerati 130 diversi progetti del corso di Programmazione ad Oggetti presentati nell'arco degli ultimi tre anni accademici e per ciascuno di questi si è effettuato il confronto con tutti i progetti a disposizione dal 2015 in poi che, complessivamente, ammontano a 354 progetti.
%
Questi sono sviluppati in Java, generalmente in \textit{team} di quattro componenti, e rappresentano artefatti di medio-alta complessità, costituiti in media da più di un centinaio di sorgenti ciascuno.

A seguito dell'analisi si è effettuata un'ispezione manuale dei sorgenti in modo tale da verificare che le stime effettuate dallo strumento non fossero fuorvianti o prive di fondamento.
%
I risultati complessivi sono presentati in \Cref{table:results}; per semplicità sono riportati solo le corrispondenze con similarità maggiore del 40\%.

\begin{table}[h!]
    \centering
    \begin{tabular}{|p{0.2\linewidth}|p{0.2\linewidth}|p{0.15\linewidth}|p{0.35\textwidth}|}
        \hline
        \textbf{Progetto originale} & \textbf{Progetto copiato} & \textbf{Similarità} & \textbf{Ispezione manuale} \\ [0.5ex] 
        \hline\hline
        gym-man & gym-man & 100\% & corrispondenza totale \\
        \hline
        spacewar & Space-Invaders & 90\% & corrispondenza totale \\
        \hline
        boxhead & zombiegame & 81\% & corrispondenza elevata \\
        \hline
        b.b.evo & paranoid & 70\% & corrispondenza medio-alta \\
        \hline
        alone-in-the-space & space-run & 62\% & corrispondenza medio-alta \\
        \hline
        SMR & Flappy Bird & 55\% & corrispondenza parziale \\
        \hline
        harbor-mgr & Teatro & 49\% & corrispondenza parziale \\
        \hline
        ALA & traffic-sim & 48\% & nessuna corrispondenza - "falso positivo" \\
        \hline
    \end{tabular}
    \caption[Risultati ottenuti confrontando i progetti sottomessi negli ultimi tre anni accademici]{Risultati ottenuti confrontando i progetti sottomessi negli ultimi tre anni accademici. Nell'ultima colonna vi è riportato l'esito dell'ispezione manuale eseguita a seguito di quella automatica; per semplicità si sono catalogate le corrispondenze in: corrispondenza \textit{totale}, \textit{elevata}, \textit{parziale} o \textit{falso positivo}.}
    \label{table:results}
\end{table}

Come si può osservare, lo strumento è in grado di rilevare molto bene copiature evidenti, ovvero casi in cui sono state copiate intere classi di sorgenti: è il caso dei primi due risultati in cui i progetti sono pressoché identici e la percentuale di similarità stimata non è inferiore al 90\%.
%
Nel secondo caso la similarità al 90\% è spiegabile dal fatto che alcune interfacce sono molto piccole e non hanno una lunghezza (in numero di \textit{token}) sufficiente per essere rilevate (cioè sono minori della minima lunghezza $s$ imposta nell'algoritmo \textit{Running Karp Rabin Greedy String Tiling}).

Come naturale che sia, invece, laddove la copiatura è effettuata in modo più scaltro, attingendo in modo parziale da altri progetti, la similarità cala, mantenendosi tuttavia al di sopra del 60\%.

Sotto il 60\% si è in presenza di codice piuttosto simile, ma in maniera molto limitata rispetto alla totalità dei sorgenti.
%
\`E il caso, ad esempio, dei progetti \texttt{SMR} e \texttt{Flappy Bird} in cui è intuibile, guardando i risultati generati dal rilevatore che gli sviluppatori del secondo abbiano preso spunto dal primo.
%
Tuttavia non è ravvisabile una copiatura evidente come nei precedenti casi.

Un solo caso, tra tutti i progetti analizzati, riporta similarità maggiore del 45\% senza essere realmente copiato.
%
Ciononostante, la similarità riportata non è superiore al 50\% e il fatto che i progetti non siano copiati né simili è facilmente verificabile in pochi minuti guardando il \textit{report} stesso.

\section{Tempi d'esecuzione}
Per quanto concerne i tempi d'esecuzione, attualmente lo strumento impiega, per fare il confronto di un progetto con i 354 progetti disponibili, grazie alla parallelizzazione e al salvataggio in locale dei sorgenti da analizzare, mediamente un terzo del tempo che avrebbe impiegato senza effettuare il \textit{caching}.

