% ! TeX root = ../../thesis.tex
\chapter{Validazione dei risultati}
\label{chapter:validation}
In questo capitolo vengono riportati e analizzati i risultati ottenuti nell'ambito di applicazione dello strumento sviluppato.

\vspace*{0.5cm}

Le prestazioni del sistema sono misurate sulla base di due aspetti fondamentali e indipendenti tra loro: le \textit{performance} in termini di tempo d'esecuzione, ma soprattutto l'accuratezza, ovvero la capacità di rilevare copiature e plagi.

\section{Accuratezza}
L'accuratezza è l'obiettivo più importante da tenere in considerazione e in relazione alla quale misurare le prestazioni.
%
Infatti, lo scopo primario è rilevare con precisione casi probabili di plagio e presentarli all'utente il più in alto possibile nei \textit{report} generati dallo strumento.
%
\`E solo in seguito ad aver raggiunto un grado di precisione soddisfacibile che possono essere adottate ottimizzazioni che permettano di ridurre il tempo di calcolo.

Dunque, per testarne l'accuratezza sono stati considerati 130 diversi progetti del corso di Programmazione ad Oggetti presentati nell'arco degli ultimi tre anni accademici e per ciascuno di questi si è effettuato il confronto con tutti i progetti a disposizione dal 2015 in poi che, complessivamente, ammontano a 354 progetti.
%
Tali progetti sono sviluppati in Java, generalmente in \textit{team} di quattro componenti, e rappresentano artefatti di medio-alta complessità, costituiti in media da più di un centinaio di sorgenti ciascuno.

A seguito dell'analisi si è effettuata un'ispezione manuale dei sorgenti in modo tale da verificare che le stime effettuate dallo strumento non fossero fuorvianti o prive di fondamento.
%
I risultati complessivi sono presentati in \Cref{table:results}.

\begin{table}[h]
    \centering
    \begin{tabular}{|p{0.15\linewidth}|p{0.15\linewidth}|p{0.15\linewidth}|p{0.45\textwidth}|}
        \hline
        \textbf{Progetto originale} & \textbf{Progetto copiato} & \textbf{Similarità} & \textbf{Ispezione manuale} \\ [0.5ex] 
        \hline\hline
        gym-man & gym-man & 100\% & I progetti sono realmente copiati e rappresentano una copia letterale uno dell'altro. \\
        \hline
        spacewar & space-invaders & 99\% & I progetti sono realmente copiati e rappresentano una copia leterale uno dell'altro. \\
        \hline
        boxhead & zombiegame & 90\% & I progetti presentano molto codice uguale. \\
        \hline
    \end{tabular}
    \caption{Risultati ottenuti confrontando i progetti sottomessi negli anni 2021, 2020, 2019 con tutti gli altri.}
    \label{table:results}
\end{table}



