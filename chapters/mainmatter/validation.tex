% ! TeX root = ../../thesis.tex
\chapter{Validazione dei risultati}
\label{chapter:validation}
In questo capitolo ci si concentra sulla validazione dei risultati ottenuti.

% \vspace*{0.5cm}

% Le prestazioni del sistema sono misurate in termini di due fattori, tra loro indipendenti: il tempo di esecuzione e la capacità di rilevamento di plagi.

% \section{Analisi dei risultati}
% Per testare la capacità di rilevamento dei plagi è stato considerato un semplice progetto Java e, passo a passo, sono state applicate modifiche incrementalmente sempre più complesse da rilevare, seguendo la tassonomia di Faidhi \& Robinson introdotta nella \Cref{01:automatic-plagiarism-detector}.
% %


% \begin{itemize}
%     \item \textbf{Livello 1}: Il progetto è lasciato inviariato, al netto di modifiche ai commenti e alla documentazione: questo livello 
%     \item \textbf{Livello 2}: In questo livello vengono cambiati gli identificatori, e i tipi di dato e le strutture dati. Questo modifiche tuttavia sono minime: per "cambio di strutture dati" si intende a questo stadio il sostituire un array con una lista o un \texttt{float} con un \texttt{double}.
%     \item \textbf{Livello 3}: In questo livello vengono applicati cambiamenti come espressioni equivalenti, ...
% \end{itemize}

% \section{Tempi di esecuzione}
