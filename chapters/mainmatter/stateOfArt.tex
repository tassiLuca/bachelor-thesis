% ! TeX root = thesis.tex
\chapter{Stato dell'arte}
\label{chapter:stateOfArt}
In questo capitolo viene riportata una panoramica delle tecniche storicamente concepite per affrontare il problema dell'individuazione delle parti di codice copiate, evidenziandone, per ciascuno, pregi e difetti.

\section{Visione d'insieme}
Come già introdotto in precedenza, la maggior parte dei sistemi automatici di rilevazione di duplicati tipicamente lavora in due fasi: una prima di \textit{tokenizzazione}, seguita da una fase di confronto.
%
Nella fase di tokenizzazione i sorgenti vengono convertiti in una rappresentazione intermedia, prima di essere confrontati nella successiva fase.

\section{Tecniche di \textit{tokenizzazione}}

\subsection{Approccio \textit{Text based}}
In questo approccio, ogni istruzione del codice sorgente viene trattata come una stringa e il programma è considerato come una mera sequenza di stringhe.
%
Questo è l'approccio, fra tutti, più fragile...

\section{}

In conclusione, bisogna rendersi conto che, a prescindere dal grado di sofisticatezza della tecnica che si utilizza, è sempre possibile che si verifichi un plagio non rilevabile.
%
Da bilanciare le risorse investite nell'individuazione del plagio e i rendimenti decrescenti di trovare i pochi, se non nessuno, casi difficili da rilevare \cite{joy-99}
