% ! TeX root = thesis.tex
\chapter{Stato dell'arte}
\label{chapter:stateOfArt}
In questo capitolo viene riportata una panoramica dei metodi che sono stati storicamente concepiti per affrontare il problema dell'individuazione delle parti di codice copiate, evidenziandone, per ciascuno, pregi e difetti.

\section{}

\section{}

In conclusione, bisogna rendersi conto che, a prescindere dal grado di sofisticatezza della tecnica che si utilizza, è sempre possibile che si verifichi un plagio non rilevabile.
%
Da bilanciare le risorse investite nell'individuazione del plagio e i rendimenti decrescenti di trovare i pochi, se non nessuno, casi difficili da rilevare \cite{joy-99}
