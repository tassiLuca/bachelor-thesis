% ! TeX root = thesis.tex
\chapter{Stato dell'arte}
\label{chapter:stateOfArt}
In questo capitolo viene riportata una panoramica delle tecniche storicamente introdotte per l'analisi del codice sorgente e l'individuazione delle parti potenzialmente copiate.

\section{Visione d'insieme}
% preso spunto da "Current trends in source code analysis, plagiarism detection and issues of analysis big datasets" -- si deve citare??
Il codice sorgente non è nient'altro che un file di testo scritto da sviluppatori, che deve essere compilato o interpretato e che, pertanto, si basa su regole sintattiche e grammaticali proprie del linguaggio di programmazione con cui è scritto che permettono a entrambi gli attori, il programmatore e il calcolatore, di "capirlo" ed elaborarlo.

Per questo motivo, se processare la struttura di un sorgente non presenta grandi difficoltà, processare il significato, ovvero l'idea e la logica sottesa al codice, costituisce una sfida più grande, se non altro perché entra in gioco la competenza dello sviluppatore e la sua esperienza nella scrittura di codice "pulito".

Poiché il problema è complesso, gli attuali metodi di elaborazione non aspirano a risolvere il problema \textit{in toto}, ma applicano il principio di decomposizione del problema e approcciano al codice sorgente da un particolare punto di vista: alcuni analizzano il codice dal punto di vista del programmatore, cercando di comprenderne il significato, altri la loro struttura.

Possiamo classificare le tecniche di analisi in tre famiglie, o livelli \cite{duracik-krsak-hrkut-2017}.

Il primo si basa sull'analisi del codice come mero testo: si presume che siano rispettate le convenzioni e il codice contenga sufficienti informazioni che ne descrivano il significato e cercano di estrarre solo queste significative informazioni aggiuntive.

Il livello successivo è simile al precedente con la differenza che non esplora il significato del testo dal punto di vista del programmatore, bensì dal punto di vista del calcolatore, che "vede" il codice come una sequenza di comandi, ciascuno con il proprio significato nel contesto della grammatica del linguaggio.

L'ultimo e terzo livello riguarda l'analisi del modello del codice sorgente.

\section{Tecniche di analisi}

\subsection{Analisi basata sul testo}
In questo approccio, ogni istruzione del codice sorgente viene trattata come una stringa e il programma è considerato come una mera sequenza di stringhe.
%
Questo è l'approccio, fra tutti, più fragile...

\subsection{Analisi basata sui \textit{token}}

\subsection{Analisi basata su un modello}

\subsection{Quale approccio scegliere?}
In conclusione, scegliere quali di questi approcci scegliere è complesso, perché difficili da confrontare tra loro: la maggior parte di queste tecniche viene valutata utilizzando il proprio \textit{set} di dati, che raramente sono resi pubblicamente accessibili \cite{karnalim-budi-toba-joy-2019}.

In conclusione, bisogna rendersi conto che, a prescindere dal grado di sofisticatezza della tecnica che si utilizza, è sempre possibile che si verifichi un plagio non rilevabile.
%
Da bilanciare le risorse investite nell'individuazione del plagio e i rendimenti decrescenti di trovare i pochi, se non nessuno, casi difficili da rilevare \cite{joy-99}.


