% ! TeX root = thesis.tex
\chapter{Introduzione}
Negli ultimi decenni, con la rapida crescita di dati e informazioni facilmente accessibili sul \textit{web}, è diventato sempre più semplice poter utilizzare, in parte o in tutto, le risorse reperite.
%
Tuttavia, l'uso improprio di tali risorse, senza attribuire i necessari crediti agli autori, costituisce, oltre che una pratica scorretta che contravviene a qualsiasi ordine deontologico, un illecito. \todo{specificare meglio: punibile a norma di legge...}

In generale, viene definito \textbf{plagio} l'atto di appropriarsi degli scritti di altre persone, in violazione della legge sul \textit{copyright} \cite{britannica}.

Anche nel mondo dell'informatica il problema del plagio è un fenomeno in crescita\todo{fonte?}, incoraggiato per lo più dalla sempre maggiore quantità di progetti \textit{software} \textit{open source}, che induce gli sviluppatori a copia incollare frammenti di codice, talvolta neppure conoscendo le relative condizioni e termini di licenza.
%
Questo induce spesso a un uso improprio del codice altrui comportando sanzioni e danni per lo sviluppatore.

Per questi motivi, già a partire dagli anni settanta del novecento, sono stati proposti algoritmi e tecniche per l'analisi del codice, nonché l'identificazione e localizzazione di plagi.

In questo contesto è da porre in evidenza la differenza tra l'individuazione di cloni e quella di plagi.
%
Quando ci si riferisce a un clone, infatti, lo si fa con riferimento a un frammento di codice che è stata copiato e marginalmente modificato. 
%
Quando invece ci si riferisce ad un plagio si intende una sezione che è stata copiata e la cui opera di copiatura si è cercato di dissimulare, mediante opportune azioni di rifattorizzazione del codice.  
%
Va da sé, quindi, che il rilevamento di \textit{cloni} mira ad individuare frammenti di codice allo scopo di evitare codice duplicato e migliorare la manutenibilità del sistema, mentre nel rilevamento di plagi lo scopo primario è identificare possibili illeciti, complicando...
%
Questo è, insieme alle prestazioni, la sfida principale da affrontare durante la progettazione di un sistema anti-plagio.

\section{Problema 1: Cambiamento di codice}
Come già anticipato nell'introduzione, la capacità del software anti-plagio di riuscire ad identificare possibili parti di codice plagiate, passa anche e soprattutto dalla capacità del sviluppatore di saper rifattorizzare il codice.

In generale, non è possibile classificare tutti i possibili metodi con cui un programma può essere trasformato in un altro, mantenendo inalterate le sue funzionalità.
%
Tuttavia, è possibile distinguere due macro categorie di modifiche: \textbf{lessicali} e \textbf{strutturali} \cite{joy-99}.

Le modifiche lessicali sono quelle che, in linea di principio, possono essere eseguite da un \textit{text editor} e non richiedono la conoscenza del linguaggio di programmazione. Alcuni esempi sono:
\begin{itemize}
    \item la riformulazione di commenti, la loro aggiunta o rimozione;
    \item la riformattazione del testo: introdurre spazi vuoti, cambiare l'ordine dei parametri nella definizione delle funzioni;
    \item cambiare gli identificatori, il nome delle funzioni o i tipi di dato...;
\end{itemize}

Le modifiche strutturali, invece, richiede un maggior sforzo in termini di capire il codice ed è fortemente dipendente dal linguaggio...

