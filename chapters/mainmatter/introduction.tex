% ! TeX root = thesis.tex
\chapter{Introduzione}
Negli ultimi decenni, con la rapida crescita di dati e informazioni facilmente accessibili sul \textit{web}, è diventato sempre più semplice poter utilizzare, in parte o in tutto, le risorse reperite.
%
Tuttavia, l'uso improprio di tali risorse, senza attribuire i necessari crediti agli autori, costituisce, oltre che una pratica scorretta che contravviene a qualsiasi ordine deontologico, un illecito. \todo{specificare meglio: punibile a norma di legge...}

In generale, l'atto di appropriarsi degli scritti di altre persone, in violazione della legge sul \textit{copyright} è definito un \textbf{plagio} \cite{britannica}.

Anche nel mondo dell'informatica il problema del plagio è un fenomeno in crescita\todo{fonte?}, incoraggiato per lo più dalla sempre maggiore quantità di progetti \textit{software} \textit{open source}, che induce gli sviluppatori a copia incollare frammenti di codice, a volte neppure conoscendo le relative condizioni e termini di licenza.
%
Questo porta spesso a un uso improprio del codice altrui che comporta sanzioni e danni per lo sviluppatore.

Nel corso degli anni si sono progettati...

Prima di addentrarci, tuttavia, nella trattazione dei metodi e delle tecniche per rilevare tali illeciti\todo{nome non bellissimo}, è bene sottolineare la differenza tra il rilevamento di \textit{cloni} e di plagi: sebbene entrambi abbiano l'obiettivo di trovare codice copiato, nei \textit{cloni} spesso lo sviluppatore copia codice terzo senza applicare sostanziali modifiche, mentre in un plagio, spesso il codice viene rifattorizzato per nascondere le sezioni copiate. 
%
Va da sé, quindi, che il rilevamento di \textit{cloni} mira ad individuare frammenti di codice allo scopo di evitare codice duplicato e migliorare la manutenibilità del sistema, mentre nel rilevamento di plagi lo scopo è individuare codice copiato per ...

