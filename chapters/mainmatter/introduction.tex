% ! TeX root = thesis.tex
\chapter{Introduzione}
Negli ultimi decenni, con la rapida crescita di dati e informazioni facilmente accessibili sul \textit{web}, è diventato sempre più semplice poter utilizzare, in parte o in tutto, le risorse reperite.
%
Tuttavia, l'uso improprio di tali risorse, senza attribuire i necessari crediti agli autori, costituisce, oltre che una pratica scorretta che contravviene a qualsiasi ordine deontologico, un illecito. \todo{specificare meglio: punibile a norma di legge...}

In generale, viene definito \textbf{plagio} l'atto di appropriarsi degli scritti di altre persone, in violazione della legge sul \textit{copyright} \cite{britannica}.

Anche nel mondo dell'informatica il problema del plagio è un fenomeno in crescita\todo{fonte?}, incoraggiato per lo più dalla sempre maggiore quantità di progetti \textit{software} \textit{open source}, che induce gli sviluppatori a copia incollare frammenti di codice, talvolta neppure conoscendo le relative condizioni e termini di licenza.
%
Questo porta spesso a un uso improprio del codice altrui arrecando sanzioni e danni per lo sviluppatore.

Già a partire dagli anni settanta del novecento, sono stati proposti algoritmi e tecniche per l'analisi del codice, nonché l'identificazione e localizzazione di plagi.

In questo contesto è da porre in evidenza la differenza tra l'individuazione di cloni e quella di plagi.
%
Quando ci si riferisce a un clone, infatti, lo si fa con riferimento a un frammento di codice che è stata copiato e marginalmente modificato. 
%
Quando invece ci si riferisce ad un plagio si intende una sezione che è stata copiata e la cui opera di copiatura si è cercato di dissimulare, mediante opportune azioni di rifattorizzazione del codice.  
%
Dunque, l'ambito di applicazione delle due ricerche, quella di cloni e quella di plagi, è nettamente diverso: se nel primo l'obiettivo è quello di evidenziare il codice duplicato al fine di migliorare la qualità del codice e migliorare la manutenibilità del sistema, nel secondo lo scopo primario è identificare possibili illeciti.

Sicuramente, questo aspetto è, insieme alle prestazioni, la sfida principale da affrontare durante la progettazione di un sistema anti-plagio.

\section{Problematiche}

\subsection{La rifattorizzazione del codice}
Come già anticipato, la capacità di un software anti-plagio nel riuscire ad identificare possibili parti di codice plagiate, passa anche e soprattutto dalla capacità dello sviluppatore di saper rifattorizzare il codice.

In generale, non è possibile classificare tutti i possibili metodi con cui un programma può essere trasformato in un altro mantenendo inalterate le sue funzionalità.
%
Tuttavia, è possibile distinguere due macro categorie di modifiche: \textbf{lessicali} e \textbf{strutturali} \cite{joy-99}.

Le modifiche lessicali sono quelle che, in linea di principio, possono essere eseguite da un \textit{text editor} e non richiedono la conoscenza del linguaggio di programmazione con cui è stato sviluppato il codice. 
%
Alcuni casi esemplificativi sono:
\begin{itemize}
    \item la riformulazione di commenti, la loro aggiunta o rimozione;
    \item la riformattazione del testo, come l'introduzione di spazi vuoti, di nuove linee o il cambio dell'ordine dei parametri nella definizione delle funzioni;
    \item cambiare il nome degli identificatori e delle funzioni o i tipi di dato: ad esempio da \texttt{int} a \texttt{Integer} o da \texttt{float} a \texttt{double}.
\end{itemize}

Le modifiche strutturali sono invece fortemente dipendenti dal linguaggio di programmazione e richiedono un maggior sforzo in termini di comprensione della logica del codice.
%
Di seguito alcuni esempi di rifattorizzazioni che rientrano in questa classe:
\begin{itemize}
    \item aggiungere istruzioni ridondanti, come dichiarazioni, inizializzazioni, istruzioni di stampa;
    \item sostituire i costrutti di loop con costrutti equivalenti: passare, ad esempio, da \texttt{for} a \texttt{do/while};
    \item sostituire istruzioni \texttt{if} nidificate con dichiarazioni equivalenti, ad esempio \texttt{when} (in Kotlin) o \texttt{switch-case}, e viceversa;
    \item cambiare l'ordine di istruzioni indipendenti;
    \item cambiare l'ordine degli operandi: ad esempio \texttt{x < y} può essere cambiato in \texttt{y >= x};
    \item sostituire la chiamata a funzione con il corpo della stessa.
\end{itemize}

Gli strumenti d'identificazione di plagi devono pertanto cercare di annullare, durante il processo di analisi, gli effetti di queste rifattorizzazioni.
%
A questo scopo le tecniche di analisi, introdotte nel \Cref{chapter:stateOfArt}, si compongono di più fasi nelle quali trasformano i sorgenti in rappresentazioni intermedie che astraggano il più possibile dai dettagli implementativi che possono essere facilmente cambiati, quindi applicano su di esse tecniche di \textit{matching}.

\subsection{Le prestazioni}

